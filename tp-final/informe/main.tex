\documentclass[11pt,a4paper]




\begin{document}

\section{Qué es un heightmap}

En computación gráfica, un heightmap es una imagen raster In computer graphics, a heightmap or heightfield is a raster image used mainly as Discrete Global Grid in secondary elevation modeling. Each pixel stores values, such as surface elevation data, for display in 3D computer graphics. A heightmap can be used in bump mapping to calculate where this 3D data would create shadow in a material, in displacement mapping to displace the actual geometric position of points over the textured surface, or for terrain where the heightmap is converted into a 3D mesh.

https://en.wikipedia.org/wiki/Heightmap

\section{Como empezamos}
Por ahora estamos tratando de crear un terreno plano
No sabiamos muy bien como encarar, asi que seguimos los pasos de este chabon https://www.youtube.com/watch?v=hHGshzIXFWY&list=PLRL3Z3lpLmH3PNGZuDNf2WXnLTHpN9hXy y usamos el esqueleto de los tps como base.

\end{document}