\documentclass[11pt,a4paper]




\begin{document}

\section{Qué es un heightmap}

En computación gráfica, un heightmap es una imagen raster In computer graphics, a heightmap or heightfield is a raster image used mainly as Discrete Global Grid in secondary elevation modeling. Each pixel stores values, such as surface elevation data, for display in 3D computer graphics. A heightmap can be used in bump mapping to calculate where this 3D data would create shadow in a material, in displacement mapping to displace the actual geometric position of points over the textured surface, or for terrain where the heightmap is converted into a 3D mesh.

https://en.wikipedia.org/wiki/Heightmap

\section{Como empezamos}
Por ahora estamos tratando de crear un terreno plano
No sabiamos muy bien como encarar, asi que seguimos los pasos de este chabon https://www.youtube.com/watch?v=hHGshzIXFWY&list=PLRL3Z3lpLmH3PNGZuDNf2WXnLTHpN9hXy y usamos el esqueleto de los tps como base.

Por algun motivo tiramos todo lo anterior (porque habia un bug sobre como unir los puntos del mesh), pasamos a hacer el terreno usando esta referencia : https://blog.mastermaps.com/2013/10/terrain-building-with-threejs.html. Esto es porque era mas explicito como hacerlo, los demas ejemplos solo asumen que uno ya sabe como hacerlo.


Cuando haciamos un terreno, este se generaba en los ejes x y, cuando en general los terrenos se generan en xz, para que y sea la altura. Intentamos rotarlo punto por punto, pero nos generaba problemas con la luz (visualmente se veian y comportaban raro).


Con este sitio arreglamos la visual de las luces https://threejsfundamentals.org/threejs/lessons/threejs-lights.html. Descubrimos el tipo de luz que queriamos usar, como funcionaba y agregamos un objeto que representaba la luz, para ganar visibilidad.
Despues descubrimos lo del rotate x, y el plano giraba correctamente y las luces se comportaron bien.


En principio las alturas decidimos generarlas de manera random, estableciendo un limite entre 0 y 5 para probar.



Ahora vamos a probar terrenos montañosos con transformaciones cuadraticas, obtenidas de este sitio http://www.stuffwithstuff.com/robot-frog/3d/hills/hill.html.

\end{document}